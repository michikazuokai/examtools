% ============================================================
% macros_v2.tex (exam)
% ============================================================
\typeout{[MACROS] macros_v2.tex LOADED}

% ---- counters ----
\newcounter{qno}
\newcounter{subno}
\newif\ifInSubquestion
\InSubquestionfalse

% ---- multiline -> choices gap fix ----
\newif\ifAfterMultiline
\AfterMultilinefalse

% まずはこの値から(あとで微調整)
\newlength{\MLtoChoiceTightNormal} \setlength{\MLtoChoiceTightNormal}{-7mm}
\newlength{\MLtoChoiceTightInline} \setlength{\MLtoChoiceTightInline}{-8mm}


% ---- layout lengths (override in styles.tex if you like) ----
\newlength{\Qlabelw} \setlength{\Qlabelw}{5mm}
\newlength{\Qsep}    \setlength{\Qsep}{1mm}
\newlength{\QsepSub} \setlength{\QsepSub}{8mm} % ←小問だけ広げたい間隔(例)
\newlength{\QBefore} \setlength{\QBefore}{1.0mm}
\newlength{\QAfter}  \setlength{\QAfter}{1.5mm}
% choices の開始位置(右寄せ量)を小問だけ増やす
\newlength{\ChoiceIndentMain} \setlength{\ChoiceIndentMain}{12mm} % ←大問 choices の開始位置(必要に応じて調整)
\newlength{\ChoiceSubExtra} \setlength{\ChoiceSubExtra}{3mm} % 小問だけ、さらに右へ(好みで調整)

\newlength{\SubIndent}      \setlength{\SubIndent}{10mm} % change later if needed
\newlength{\SubBlockBefore} \setlength{\SubBlockBefore}{1.0mm}
\newlength{\SubBlockAfter}  \setlength{\SubBlockAfter}{1.0mm}
% question本文の開始位置(番号+隙間)
\newlength{\QBodyIndent} \setlength{\QBodyIndent}{0pt}
\newlength{\ChoiceIndentTmp} % choices 用の一時長さ(1回だけ定義)

% sline/multiline の開始位置(大問/小問で切替)
\newlength{\TextIndentMain} \setlength{\TextIndentMain}{8mm} % 大問のsline/multiline開始位置
\newlength{\TextIndentSubExtra} \setlength{\TextIndentSubExtra}{4mm} % 小問だけ追加で右へ(好みで調整)
\newlength{\TextIndentTmp} % 一時用(毎回newlengthしない)

% --- labelと本文の間隔(追加) ---
\newlength{\ChoiceLabelSepInline}
\newlength{\ChoiceLabelSepNormal}

% 初期値(好みで調整)
\setlength{\ChoiceLabelSepInline}{4mm} % inlineは少し広め推奨
\setlength{\ChoiceLabelSepNormal}{4mm} % normalは控えめでもOK


\newcommand{\DBGvsp}[1]{%
  \typeout{[DBG] #1 : vmode=\ifvmode YES\else NO\fi, parskip=\the\parskip, lastskip=\the\lastskip, baselineskip=\the\baselineskip}%
}



% ---- pgfkeys family ----
\pgfkeys{/exam/.is family, /exam,
  type/.initial=normal,
  sep/.initial=8,         % mm
  cols/.initial=4, % ★追加
  before/.initial=0,      % mm
  after/.initial=0,       % mm
  % NOTE: makelatex emits linenumber=true|false (string).
  % We also accept 1/0 for backward compatibility.
  linenumber/.initial=false,
}

\pgfkeys{/exam/vspace/.initial=0mm}

% ---- helpers ----
\NewDocumentCommand{\ChoiceLabel}{m}{#1} % customize: (A) / A) / etc.

% makelatex emits \citem{A}{...}. Use one canonical item macro.
%\NewDocumentCommand{\citem}{m m}{\item[\ChoiceLabel{#1}] #2}
\NewDocumentCommand{\citem}{m m}{\item[\ChoiceLabel{#1}]#2}


% ============================================================
% Cover page
%  \ExamCover{<title>}{<notes>}[<footer>]
% - cover page has NO page number
% - after cover: page numbering starts from 1 (arabic)
% ============================================================
\NewDocumentCommand{\ExamCover}{m m O{}}{%
  \gdef\ExamTitle{#1}% ★追加:第1引数をフッター用に保存(global)
  \clearpage
  \begingroup
    \pagestyle{empty}%
    \thispagestyle{empty}

    \begin{tikzpicture}[remember picture,overlay]
      % 例:ページ上部からの固定位置にタイトル枠を置く yshiftで上下の位置指定
      \node[
        draw, line width=0.6pt,
        minimum width=15cm, minimum height=3.5cm,
        align=center, inner sep=0pt,
        anchor=north
      ] at ([yshift=-25mm]current page.north) {%
        {\Huge #1}\\{\Huge 履修判定試験}%
      };

      % 例:同様に注意事項枠を固定位置へ
      \node[
        draw, line width=0.6pt,
        minimum width=15.5cm, minimum height=4.5cm,
        align=left, inner sep=6mm,
        anchor=north
      ] at ([yshift=-120mm]current page.north) {%
        \begin{minipage}{15cm}
          \raggedright
          {\large
            \begin{itemize}
              \setlength{\itemsep}{0.3em}
              \setlength{\topsep}{0pt}
              #2%
            \end{itemize}
          }%
        \end{minipage}
      };

      % 学校名(下寄せ例)
      \node[align=center] at ([yshift=18mm]current page.south) {%
        {\Large 専門学校}\\{\Large 東京テクニカルカレッジ}%
      };
    \end{tikzpicture}

  \endgroup


  % ===== Cover page 2 (footer only) =====
  \clearpage
  \begingroup
    \pagestyle{empty}%
    \thispagestyle{empty}%

    % 何もないとページが出ないので最小の中身
    \null\vfill

    % 右下に #3 だけ出す(ページ番号は出ない)
%    \noindent\hfill{\small\ttfamily #3}\par
    \noindent\hfill{\small\ttfamily\color{black!45} #3}\par
  \endgroup

  \clearpage
  \pagestyle{fancy}%
  \setcounter{page}{1}%
}

% ---- question env ----
\NewDocumentEnvironment{question}{m}{%
  \par\vspace*{\QBefore}%
  \ifInSubquestion
    \stepcounter{subno}%
    \edef\examLabelTmp{\theqno-\thesubno}%
    \def\examSepTmp{\QsepSub}% ←小問は広い
  \else
    \stepcounter{qno}%
    \setcounter{subno}{0}%
    \edef\examLabelTmp{\theqno}%
    \def\examSepTmp{\Qsep}% ←大問は従来
  \fi
  \noindent
  \setlength{\QBodyIndent}{\dimexpr\Qlabelw+\examSepTmp\relax}%
  \makebox[\Qlabelw][l]{\textbf{\examLabelTmp}}%
  \hspace{\examSepTmp}%
  \parbox[t]{\dimexpr\linewidth-\Qlabelw-\examSepTmp\relax}{\textbf{#1}}%
  \par\vspace*{\QAfter}%
}{%
  \global\AfterMultilinefalse % ★これを追加:次の問題へ持ち越さない
  \par\vspace{0.8\baselineskip}
}

% ---- subquestion env ----
\NewDocumentEnvironment{subquestion}{}{%
  \par\vspace*{\SubBlockBefore}%
  \begingroup
  \InSubquestiontrue
%%  \begin{list}{}{\setlength{\leftmargin}{\SubIndent}\setlength{\rightmargin}{0pt}}
  \begin{list}{}{%
    \setlength{\leftmargin}{\SubIndent}%
    \setlength{\rightmargin}{0pt}%
    \setlength{\topsep}{0pt}%      ★追加:subquestion直後の上余白
    \setlength{\partopsep}{0pt}%   ★追加:段落直後の追加余白
    \setlength{\parsep}{0pt}%      ★追加
    \setlength{\itemsep}{0pt}%     ★追加
  }
  \item[]
}{%
  \end{list}
  \InSubquestionfalse
  \endgroup
  \global\AfterMultilinefalse % ★追加:小問ブロックを跨いで誤爆しないようにする
  \par\vspace*{\SubBlockAfter}%
}

% ---- sline ----
\NewDocumentCommand{\sline}{O{} m}{%
  \pgfkeys{/exam, before=0, after=0, #1}%
  \edef\SLBefore{\pgfkeysvalueof{/exam/before}}%
  \edef\SLAfter{\pgfkeysvalueof{/exam/after}}%

  % インデント決定(大問/小問)
  \ifInSubquestion
    \setlength{\TextIndentTmp}{\QBodyIndent}%
    \addtolength{\TextIndentTmp}{\TextIndentSubExtra}%
  \else
    \setlength{\TextIndentTmp}{\TextIndentMain}%
  \fi

  \ifdim\SLBefore mm=0mm\else\vspace*{\SLBefore mm}\fi%
  \noindent\hspace*{\TextIndentTmp}#2\par%
  \ifdim\SLAfter mm=0mm\else\vspace*{\SLAfter mm}\fi%
}

% ---- multiline ----
\NewDocumentEnvironment{multiline}{O{}}{%
  \pgfkeys{/exam, before=0, after=0, #1}%
  \edef\MLBefore{\pgfkeysvalueof{/exam/before}}%
  \edef\MLAfter{\pgfkeysvalueof{/exam/after}}%

  % インデント決定(大問/小問)
  \ifInSubquestion
    \setlength{\TextIndentTmp}{\QBodyIndent}%
    \addtolength{\TextIndentTmp}{\TextIndentSubExtra}%
  \else
    \setlength{\TextIndentTmp}{\TextIndentMain}%
  \fi

  \ifdim\MLBefore mm=0mm\else\vspace*{\MLBefore mm}\fi%
  \begingroup
  \setlength{\parindent}{0pt}%
  \leftskip=\TextIndentTmp % ←複数行をまとめて右寄せ
}{%
  \par % 段落を確定
  \endgroup
  \global\AfterMultilinetrue % ★「次がchoicesなら補正してね」のフラグ
  \DBGvsp{ML-end (after endgroup)}%
%%  \vspace*{-5.0mm}% ★ 少し上に引っ張って隙間を殺す
  \ifdim\MLAfter mm=0mm\else\vspace*{\MLAfter mm}\fi%
}

\NewDocumentCommand{\mline}{m}{#1\par}

% ---- image ----
\NewDocumentCommand{\image}{m m}{%
  \begin{center}
    \includegraphics[width=#2\textwidth]{#1}%
  \end{center}
}

% % ---- code ----
% % IMPORTANT: listings is a verbatim-like environment.
% % Do NOT wrap \begin{lstlisting} inside \NewDocumentEnvironment; it causes
% % "Emergency stop" / end-scanning failures.
% % Use \lstnewenvironment so the body is treated as listings verbatim.

% ---- safe listdepth getter for logging ----
\makeatletter
\newcommand{\DBGlistdepth}{\number\@listdepth} % ← \the\@listdepth でも可
\makeatother

% ---- choices debug helper ----
\newcommand{\DBGchoices}[1]{%
  \typeout{[CHO-DBG] #1 : line=\number\inputlineno, InSubq=\ifInSubquestion TRUE\else FALSE\fi, %
    listdepth=\DBGlistdepth, vmode=\ifvmode YES\else NO\fi, lastskip=\the\lastskip}%
}




\makeatletter
\@ifundefined{code}{%
  \lstnewenvironment{code}[1][]{%
    \begingroup%
    \pgfkeys{/exam, linenumber=false, #1}%
    \edef\examln{\pgfkeysvalueof{/exam/linenumber}}%
    \def\examtrue{true}%
    \def\examone{1}%
    \ifx\examln\examtrue%
      \lstset{numbers=left, numberstyle=\footnotesize, stepnumber=1, numbersep=8pt}%
    \else\ifx\examln\examone%
      \lstset{numbers=left, numberstyle=\footnotesize, stepnumber=1, numbersep=8pt}%
    \else%
      \lstset{numbers=none}%
    \fi\fi%
  }{%
    \endgroup%
  }%
}{}%
\makeatother


% ==============================
% choices (single env, robust)
% - type=normal : vertical
% - type=inline : horizontal (enumitem itemize*)
% ==============================
% --- choices indent tuning (1回だけ定義) ---
\newlength{\ChoiceInlineAdjustMain} % 大問 inline を左右に微調整(負なら左)
\newlength{\ChoiceNormalAdjustSub}  % 小問 normal を右へ微調整(正なら右)

% 初期値(まずはこのくらいから試す)
\setlength{\ChoiceInlineAdjustMain}{-3mm} % 大問 inline を少し左へ
\setlength{\ChoiceNormalAdjustSub}{3mm}   % 小問 normal を少し右へ

% choices内部で使う作業用
\newlength{\ChoiceIndentInline}
\newlength{\ChoiceIndentNormal}

% ---- choices ----
\NewDocumentEnvironment{choices}{O{}}{%
  \begingroup

  % keys
  \pgfkeys{/exam/.cd, type=normal, sep=8, vspace=3mm, #1}%
  \edef\examChoiceType{\pgfkeysvalueof{/exam/type}}%
  \edef\examChoiceSep{\pgfkeysvalueof{/exam/sep}}%
  \edef\examChoiceVspace{\pgfkeysvalueof{/exam/vspace}}%

  % ★ 直前が multiline のときだけ、余計な縦間隔を打ち消す(影響範囲を局所化)
  \ifAfterMultiline
    \global\AfterMultilinefalse
    \par\nointerlineskip
    \ifdefstring{\examChoiceType}{inline}{%
      \vspace*{\MLtoChoiceTightInline}%
      \typeout{[ML2C] applied inline tighten=\the\MLtoChoiceTightInline}%
    }{%
      \vspace*{\MLtoChoiceTightNormal}%
      \typeout{[ML2C] applied normal tighten=\the\MLtoChoiceTightNormal}%
    }%
  \fi

  % % ★ choices の前に空き(縦横どちらにも効く)
  % \par\vspace*{\examChoiceVspace}%
  % ★ choices の前に空き(積み上がりを防ぐ)
  \DBGvsp{CHOICES-begin (before addvspace)}%
  \par\addvspace{\examChoiceVspace}% ←★ここを vspace* から addvspace に
  \DBGvsp{CHOICES-begin (after addvspace)}%

  % ★ まず「共通の基準インデント」を決める(大問/小問)
  \setlength{\ChoiceIndentTmp}{\ChoiceIndentMain}%
  \ifInSubquestion
    \addtolength{\ChoiceIndentTmp}{\ChoiceSubExtra}%
  \fi

  % ★ 表示用インデントを分ける(normal用 / inline用)
  \setlength{\ChoiceIndentNormal}{\ChoiceIndentTmp}%
  \setlength{\ChoiceIndentInline}{\ChoiceIndentTmp}%

  % ★ 要望に合わせて“条件付き補正”
  %   - 大問:inline を少し左へ(大問 normal と合わせる)
  %   - 小問:normal を少し右へ(小問 inline と合わせる)
  \ifInSubquestion
    \addtolength{\ChoiceIndentNormal}{\ChoiceNormalAdjustSub}%
  \else
    \addtolength{\ChoiceIndentInline}{\ChoiceInlineAdjustMain}%
  \fi

  \typeout{[choicesxx] InSubq=\ifInSubquestion TRUE\else FALSE\fi, %
    Tmp=\the\ChoiceIndentTmp, Normal=\the\ChoiceIndentNormal, Inline=\the\ChoiceIndentInline}%
  \typeout{[choices] type=\examChoiceType, sep=\examChoiceSep, vspace=\examChoiceVspace}%

  % ---- inline / normal ----
  \ifdefstring{\examChoiceType}{inline}{%
    % ★ inline は minipage を右にずらして「箱ごと」インデント(確実)
    % \par\noindent
    \noindent % ←★ここを \par\noindent から \noindent に
    \hspace*{\ChoiceIndentInline}%
    \begin{minipage}[t]{\dimexpr\linewidth-\ChoiceIndentInline\relax}

    \begin{itemize*}[
      label={},
      leftmargin=0pt,
    %  labelsep=\ChoiceLabelSepInline, % ★追加:ラベルと本文の間隔
      afterlabel=\hspace*{\ChoiceLabelSepInline}, % ★ここが効く
      topsep=0pt,       % ★追加
      partopsep=0pt,    % ★追加(段落直後の余白を殺す)
      parsep=0pt,       % ★追加
      itemjoin=\hspace*{\examChoiceSep mm},
      itemjoin*=\hspace*{\examChoiceSep mm},
      before=\vspace{0mm},
      after=\vspace{0mm}
    ]%
    \DBGchoices{INLINE (just after begin itemize*)}%
  %  \RenewDocumentCommand{\citem}{m m}{\item \ChoiceLabel{##1}\ ##2}%
  %   \RenewDocumentCommand{\citem}{m m}{\item \ChoiceLabel{##1}\hspace{\ChoiceLabelSepInline}##2}%
  %  \RenewDocumentCommand{\citem}{m m}{\item[\ChoiceLabel{##1}] ##2}%
  %   \RenewDocumentCommand{\citem}{m m}{\item \ChoiceLabel{##1}\kern\ChoiceLabelSepInline ##2}%
  }{%
    % 縦(通常)
    \begin{itemize}[
      label={},
      leftmargin=\ChoiceIndentNormal,
      labelsep=\ChoiceLabelSepNormal,
      itemsep=0.6mm,
      partopsep=0pt,    % ★追加(段落直後の余白を殺す)
      parsep=0pt,       % ★追加      
      topsep=0.6mm
    ]%
    \DBGchoices{NORMAL (just after begin itemize)}%
  }%
}{%
  \ifdefstring{\examChoiceType}{inline}{%
    \end{itemize*}%
    \end{minipage}%
  }{%
    \end{itemize}%
  }%
  \endgroup
}

