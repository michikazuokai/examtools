% ============================================================
% macros_v2.tex (exam)
% ============================================================

% ---- counters ----
\newcounter{qno}
\newcounter{subno}
\newif\ifInSubquestion
\InSubquestionfalse

% ---- layout lengths (override in styles.tex if you like) ----
\newlength{\Qlabelw} \setlength{\Qlabelw}{5mm}
\newlength{\Qsep}    \setlength{\Qsep}{1mm}
\newlength{\QsepSub} \setlength{\QsepSub}{8mm} % ←小問だけ広げたい間隔(例)
\newlength{\QBefore} \setlength{\QBefore}{1.0mm}
\newlength{\QAfter}  \setlength{\QAfter}{1.5mm}
% choices の開始位置(右寄せ量)を小問だけ増やす
\newlength{\ChoiceIndentMain} \setlength{\ChoiceIndentMain}{12mm} % ←大問 choices の開始位置(必要に応じて調整)
\newlength{\ChoiceSubExtra} \setlength{\ChoiceSubExtra}{3mm} % 小問だけ、さらに右へ(好みで調整)

\newlength{\SubIndent}      \setlength{\SubIndent}{10mm} % change later if needed
\newlength{\SubBlockBefore} \setlength{\SubBlockBefore}{1.0mm}
\newlength{\SubBlockAfter}  \setlength{\SubBlockAfter}{1.0mm}
% question本文の開始位置(番号+隙間)
\newlength{\QBodyIndent} \setlength{\QBodyIndent}{0pt}
\newlength{\ChoiceIndentTmp} % choices 用の一時長さ(1回だけ定義)

% sline/multiline の開始位置(大問/小問で切替)
\newlength{\TextIndentMain} \setlength{\TextIndentMain}{8mm} % 大問のsline/multiline開始位置
\newlength{\TextIndentSubExtra} \setlength{\TextIndentSubExtra}{4mm} % 小問だけ追加で右へ(好みで調整)
\newlength{\TextIndentTmp} % 一時用(毎回newlengthしない)

% ---- pgfkeys family ----
\pgfkeys{/exam/.is family, /exam,
  type/.initial=normal,
  sep/.initial=8,         % mm
  cols/.initial=4, % ★追加
  before/.initial=0,      % mm
  after/.initial=0,       % mm
  % NOTE: makelatex emits linenumber=true|false (string).
  % We also accept 1/0 for backward compatibility.
  linenumber/.initial=false,
}

% ---- helpers ----
\NewDocumentCommand{\ChoiceLabel}{m}{#1} % customize: (A) / A) / etc.

% makelatex emits \citem{A}{...}. Use one canonical item macro.
\NewDocumentCommand{\citem}{m m}{\item[\ChoiceLabel{#1}] #2}

% ============================================================
% Cover page
%  \ExamCover{<title>}{<notes>}[<footer>]
% - cover page has NO page number
% - after cover: page numbering starts from 1 (arabic)
% ============================================================

\NewDocumentCommand{\ExamCover}{m m O{}}{%
  \clearpage
  \begingroup
    \pagestyle{empty}%
    \thispagestyle{empty}
    \begin{center}

      \vspace*{16mm}

      % ----------------------------
      % 文字(普通幅)→ 後から枠を描く(15cm x 3.5cm)
      % ----------------------------
      \begin{tikzpicture}[remember picture]
        \node (C) [align=center] {%
          {\Huge #1}\\
          {\Huge 履修判定試験}%
        };
      \end{tikzpicture}%
      \begin{tikzpicture}[remember picture,overlay]
        \draw[line width=0.6pt]
          ($(C.center)+(-7.25cm,-1.75cm)$) rectangle
          ($(C.center)+( 7.25cm, 1.75cm)$);
      \end{tikzpicture}

      \vspace{75mm}

      \begin{tikzpicture}[remember picture]
        % 先に中身(文字)を出す:文字間隔は普通のまま
        \node (N) [align=left] {%
          \begin{minipage}{15cm}
            \raggedright
            {\large
              \begin{itemize}
                \setlength{\itemsep}{0.3em}
                \setlength{\topsep}{0pt}
                #2%
              \end{itemize}
            }%
          \end{minipage}
        };
      \end{tikzpicture}%
      \begin{tikzpicture}[remember picture,overlay]
        % あとから枠だけ描く:15cm×3.5cm固定
        \draw[line width=0.6pt]
          ($(N.center)+(-7.75cm,-2.25cm)$) rectangle
          ($(N.center)+( 7.75cm, 2.25cm)$);
      \end{tikzpicture}

      \vfill
      {\Large 専門学校\par}
      {\Large 東京テクニカルカレッジ\par}

    \end{center}
  \endgroup

  % ===== Cover page 2 (footer only) =====
  \clearpage
  \begingroup
    \pagestyle{empty}%
    \thispagestyle{empty}%

    % 何もないとページが出ないので最小の中身
    \null\vfill

    % 右下に #3 だけ出す(ページ番号は出ない)
%    \noindent\hfill{\small\ttfamily #3}\par
    \noindent\hfill{\small\ttfamily\color{black!45} #3}\par
  \endgroup

  \clearpage
  \pagestyle{fancy}%
  \setcounter{page}{1}%
}

% ---- question env ----
\NewDocumentEnvironment{question}{m}{%
  \par\vspace*{\QBefore}%
  \ifInSubquestion
    \stepcounter{subno}%
    \edef\examLabelTmp{\theqno-\thesubno}%
    \def\examSepTmp{\QsepSub}% ←小問は広い
  \else
    \stepcounter{qno}%
    \setcounter{subno}{0}%
    \edef\examLabelTmp{\theqno}%
    \def\examSepTmp{\Qsep}% ←大問は従来
  \fi
  \noindent
  \setlength{\QBodyIndent}{\dimexpr\Qlabelw+\examSepTmp\relax}%
  \makebox[\Qlabelw][l]{\textbf{\examLabelTmp}}%
  \hspace{\examSepTmp}%
  \parbox[t]{\dimexpr\linewidth-\Qlabelw-\examSepTmp\relax}{\textbf{#1}}%
  \par\vspace*{\QAfter}%
}{%
  \par\vspace{0.8\baselineskip}
}

% ---- subquestion env ----
\NewDocumentEnvironment{subquestion}{}{%
  \par\vspace*{\SubBlockBefore}%
  \begingroup
  \InSubquestiontrue
  \begin{list}{}{\setlength{\leftmargin}{\SubIndent}\setlength{\rightmargin}{0pt}}
  \item[]
}{%
  \end{list}
  \InSubquestionfalse
  \endgroup
  \par\vspace*{\SubBlockAfter}%
}

% ---- sline ----
\NewDocumentCommand{\sline}{O{} m}{%
  \pgfkeys{/exam, before=0, after=0, #1}%
  \edef\SLBefore{\pgfkeysvalueof{/exam/before}}%
  \edef\SLAfter{\pgfkeysvalueof{/exam/after}}%

  % インデント決定(大問/小問)
  \ifInSubquestion
    \setlength{\TextIndentTmp}{\QBodyIndent}%
    \addtolength{\TextIndentTmp}{\TextIndentSubExtra}%
  \else
    \setlength{\TextIndentTmp}{\TextIndentMain}%
  \fi

  \ifdim\SLBefore mm=0mm\else\vspace*{\SLBefore mm}\fi%
  \noindent\hspace*{\TextIndentTmp}#2\par%
  \ifdim\SLAfter mm=0mm\else\vspace*{\SLAfter mm}\fi%
}

% ---- multiline ----
\NewDocumentEnvironment{multiline}{O{}}{%
  \pgfkeys{/exam, before=0, after=0, #1}%
  \edef\MLBefore{\pgfkeysvalueof{/exam/before}}%
  \edef\MLAfter{\pgfkeysvalueof{/exam/after}}%

  % インデント決定(大問/小問)
  \ifInSubquestion
    \setlength{\TextIndentTmp}{\QBodyIndent}%
    \addtolength{\TextIndentTmp}{\TextIndentSubExtra}%
  \else
    \setlength{\TextIndentTmp}{\TextIndentMain}%
  \fi

  \ifdim\MLBefore mm=0mm\else\vspace*{\MLBefore mm}\fi%
  \begingroup
  \setlength{\parindent}{0pt}%
  \leftskip=\TextIndentTmp % ←複数行をまとめて右寄せ
}{%
  \endgroup
  \ifdim\MLAfter mm=0mm\else\vspace*{\MLAfter mm}\fi%
}

\NewDocumentCommand{\mline}{m}{#1\par}

% ---- image ----
\NewDocumentCommand{\image}{m m}{%
  \begin{center}
    \includegraphics[width=#2\textwidth]{#1}%
  \end{center}
}

% % ---- code ----
% % IMPORTANT: listings is a verbatim-like environment.
% % Do NOT wrap \begin{lstlisting} inside \NewDocumentEnvironment; it causes
% % "Emergency stop" / end-scanning failures.
% % Use \lstnewenvironment so the body is treated as listings verbatim.

\makeatletter
\@ifundefined{code}{%
  \lstnewenvironment{code}[1][]{%
    \begingroup%
    \pgfkeys{/exam, linenumber=false, #1}%
    \edef\examln{\pgfkeysvalueof{/exam/linenumber}}%
    \def\examtrue{true}%
    \def\examone{1}%
    \ifx\examln\examtrue%
      \lstset{numbers=left, numberstyle=\footnotesize, stepnumber=1, numbersep=8pt}%
    \else\ifx\examln\examone%
      \lstset{numbers=left, numberstyle=\footnotesize, stepnumber=1, numbersep=8pt}%
    \else%
      \lstset{numbers=none}%
    \fi\fi%
  }{%
    \endgroup%
  }%
}{}%
\makeatother


% ==============================
% choices (single env, robust)
% - type=normal : vertical
% - type=inline : horizontal (enumitem itemize*)
% ==============================

% ---- choices ----
\NewDocumentEnvironment{choices}{O{}}{%
  \begingroup
  \pgfkeys{/exam, type=normal, sep=8, #1}%
  \edef\examChoiceType{\pgfkeysvalueof{/exam/type}}%
  \edef\examChoiceSep{\pgfkeysvalueof{/exam/sep}}%

  % ★ choices の開始位置(基本は本文開始位置に合わせる)
  \setlength{\ChoiceIndentTmp}{\QBodyIndent}%
  % ★ 小問だけ追加で右へ
  \ifInSubquestion
    \setlength{\ChoiceIndentTmp}{\QBodyIndent}%
    \addtolength{\ChoiceIndentTmp}{\ChoiceSubExtra}%
  \else
    \setlength{\ChoiceIndentTmp}{\ChoiceIndentMain}%
  \fi

  \ifstrequal{\examChoiceType}{inline}{%
    \begin{itemize*}[
      label={},
      leftmargin=\ChoiceIndentTmp,
      itemjoin=\hspace*{\examChoiceSep mm},
      itemjoin*=\hspace*{\examChoiceSep mm},
      before=\vspace{0mm},
      after=\vspace{0mm}
    ]%
  }{%
    \begin{itemize}[
      label={},
      leftmargin=\ChoiceIndentTmp,
      itemsep=0.6mm,
      topsep=0.6mm
    ]%
  }%
}{%
  \ifstrequal{\examChoiceType}{inline}{\end{itemize*}}{\end{itemize}}%
  \endgroup
}